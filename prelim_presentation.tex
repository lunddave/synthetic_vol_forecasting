\documentclass{beamer}

% Theme choice:
\usetheme{CambridgeUS}

%Packages
\usepackage[backend=biber,hyperref=true,doi=false,url=false,isbn=false, uniquename=false, uniquelist=false, style = authoryear-comp]{biblatex}
%https://www.overleaf.com/learn/latex/Biblatex_citation_styles
\addbibresource{synthVolForecast.bib}
%\usepackage{bibentry}

% \usepackage{graphicx}
% \usepackage{amsmath}
% \usepackage{amsfonts}
% \usepackage{amsthm}
% \usepackage[export]{adjustbox}
% \usepackage{amssymb}
% \usepackage[useregional]{datetime2}
% \usepackage{verbatim}
% \usepackage{mathtools}% http://ctan.org/pkg/mathtools
% \usepackage{mathrsfs}

% % use packages
% \usepackage{amscd}
% \usepackage{url}
% %\usepackage[table,xcdraw,usenames]{xcolor}
% %\usepackage[usenames]{color}

% \usepackage{subcaption}
% \usepackage{enumitem}
% \usepackage{authblk}
% \usepackage{bm}
% \usepackage{comment}
% \usepackage{pdfpages}

% \usepackage{hyperref}
% \usepackage{caption}
% \usepackage{float}
% %\usepackage[caption = false]{subfig}
% \usepackage{tikz}
% \usepackage{multirow}
% \usepackage[linesnumbered, ruled,vlined]{algorithm2e}
% \usepackage{pdflscape}
% \usepackage{etoolbox}

% %\AtBeginEnvironment{align}{\setcounter{equation}{0}} % https://tex.stackexchange.com/questions/349247/how-do-i-reset-the-counter-in-align

% % function definition
% \newcommand{\ret}{\textbf{r}}
% \newcommand{\y}{\textbf{y}}
% \newcommand{\w}{\textbf{w}}
% \newcommand{\x}{\textbf{x}}
% \newcommand{\dbf}{\textbf{d}}
% \newcommand{\X}{\textbf{X}}
% \newcommand{\Y}{\textbf{Y}}
% %\newcommand{\L}{\textbf{L}}
% \newcommand{\Hist}{\mathcal{H}}
% \newcommand{\Prob}{\mathbb{P}}
% \def\mbf#1{\mathbf{#1}} % bold but not italic
% \def\ind#1{\mathrm{1}(#1)} % indicator function
% \newcommand{\simiid}{\stackrel{iid}{\sim}} %[] IID 
% \def\where{\text{ where }} % where
% \newcommand{\indep}{\perp \!\!\! \perp } % independent symbols
% \def\cov#1#2{\mathrm{Cov}(#1, #2)} % covariance 
% \def\mrm#1{\mathrm{#1}} % remove math
% \newcommand{\reals}{\mathbb{R}} % Real number symbol
% \def\t#1{\tilde{#1}} % tilde
% \def\normal#1#2{\mathcal{N}(#1,#2)} % normal
% \def\mbi#1{\boldsymbol{#1}} % Bold and italic (math bold italic)
% \def\v#1{\mbi{#1}} % Vector notation
% \def\mc#1{\mathcal{#1}} % mathical
% \DeclareMathOperator*{\argmax}{arg\,max} % arg max
% \DeclareMathOperator*{\argmin}{arg\,min} % arg min
% \def\E{\mathbb{E}} % Expectation symbol
% \def\mc#1{\mathcal{#1}}
% \def\var#1{\mathrm{Var}(#1)} % Variance symbol
% \def\checkmark{\tikz\fill[scale=0.4](0,.35) -- (.25,0) -- (1,.7) -- (.25,.15) -- cycle;} % checkmark
% \newcommand\red[1]{{\color{red}#1}}
% \def\bs#1{\boldsymbol{#1}}
% \def\P{\mathbb{P}}
% \def\var{\mathbf{Var}}
% \def\naturals{\mathbb{N}}
% \def\cp{\overset{p}{\to}}
% \def\clt{\overset{\mathcal{L}^2}{\to}}

% \setcounter{tocdepth}{4}
% \setcounter{secnumdepth}{4}

% \newcommand{\ceil}[1]{\lceil #1 \rceil}
% \newcommand{\norm}[1]{\left\lVert#1\right\rVert} % A norm with 1 argument
% \DeclareMathOperator{\Var}{Var} % Variance symbol

% \newtheorem{cor}{Corollary}
% \newtheorem{lem}{Lemma}
% \newtheorem{thm}{Theorem}
% \newtheorem{defn}{Definition}
% \newtheorem{prop}{Proposition}
% \theoremstyle{definition}
% \newtheorem{remark}{Remark}
% \hypersetup{
%   linkcolor  = blue,
%   citecolor  = blue,
%   urlcolor   = blue,
%   colorlinks = true,
% } % color setup

% % \makeatletter
% % \setbeamertemplate{footline}
% % {
% %     \leavevmode%
% %     \hbox{%
% %         \begin{beamercolorbox}[wd=.333333\paperwidth,ht=2.25ex,dp=1ex,center]{author in head/foot}%
% %             \usebeamerfont{author in head/foot}\insertshortauthor
% %         \end{beamercolorbox}%
% %         \begin{beamercolorbox}[wd=.333333\paperwidth,ht=2.25ex,dp=1ex,center]{title in head/foot}%
% %             \usebeamerfont{title in head/foot}\insertshorttitle
% %         \end{beamercolorbox}%
% %         \begin{beamercolorbox}[wd=.333333\paperwidth,ht=2.25ex,dp=1ex,right]{date in head/foot}%
% %             \usebeamerfont{date in head/foot}\insertshortdate{}\hspace*{2em}
% %             \insertframenumber{} / \inserttotalframenumber\hspace*{2ex} 
% %         \end{beamercolorbox}}%
% %         \vskip0pt%
% %     }
% %     \makeatother

\title{Synthetic Volatility Forecasting and Other Aggregation Techniques for Time Series Forecasting}
\subtitle{Preliminary Exam}
\author{David Lundquist\thanks{davidl11@ilinois.edu}}
\date{\today}

\begin{document}

%% title frame
\begin{frame}
\titlepage
\end{frame}

\section{Introduction}

\begin{frame}
\frametitle{A seemingly unprecedented event might provoke the questions}
\begin{enumerate}
    \item What does it resemble from the past?
    \item What past events are most relevant?
    \item Can we incorporate past events in a systematic, principled manner? 
\end{enumerate}
\end{frame}

\begin{frame}
    \frametitle{When would we ever have to do this?}

    \begin{itemize}
        \item Event-driven investing strategies (unscheduled news shock) 
        \item Structural shock to macroeconomic conditions (scheduled news possibly pre-empted by news shock)
        \item Biomedical panel data subject to exogenous shock or interference
    \end{itemize}

Example: weekend of March 7th and 8th, 2020

\end{frame}

\begin{frame}
\frametitle{Punchline of the paper}

Forecasting is possible under structural shocks, so long as we incorporate external information to account for the nonzero errors.

\end{frame}

\begin{frame}
    \frametitle{Background and related methods}
    Volatility Modeling

    \begin{itemize}
        \item GARCH is slow to react \parencite[][]{andersen2003modeling}
        \item Asymmetric GARCH models may react faster but need post-shock data 
        \item Realized GARCH \parencite[][]{hansen2012realized}, in our setting, would require post-shock information and/or high-frequency data in order to outperform, and the model is highly parameterized 
    \end{itemize}

    Forecast Augmentation
    \begin{itemize}
        \item \cite[][]{clements1996intercept,clements1998forecasting} laid the groundwork for modeling nonzero errors in time series forecasting
        \item \cite[][]{guerron2017macroeconomic} use a series' own errors to correct the forecast for that series
        \item \cite[][]{dendramis2020similarity} use a similarity-based procedure to correct linear parameters in time series forecasts
        \item \cite[][]{foroni2022forecasting} adjust pandemic-era forecasts using intercept correction techniques and data from Great Financial Crisis
    \end{itemize} 
\end{frame}

% Outline frame
\begin{frame}{Outline}
    \tableofcontents
\end{frame}

% Presentation structure
\section{Setting}

% Setting for the problem
\begin{frame}
\frametitle{The news has broken but markets are closed}

\begin{itemize}
\item $y\in \mathbb{R}^{n}$, a mean-zero, real-valued response to be predicted 

\end{itemize}
\end{frame}

\begin{frame}
    \frametitle{A Primer on GARCH}
    %     \sigma^{2}_{i,t} = \omega_{i} + \omega^{*}_i + \sum^{m_{i}}_{k=1}\alpha_{i,k}a^{2}_{i,t-k} + \sum_{j=1}^{s_{i}}\beta_{i,j}\sigma_{i,t-j}^{2} + \gamma_{i}^{T} \x_{i,t} \text{ }\\[.2cm]
    %    a_{i,t} = \sigma_{i,t}(\epsilon_{i,t}(1-D^{level}_{i,t}) + \epsilon^{*}_{i}D^{level}_{i,t})\\[.2cm]
    %   \textcolor{red}{\omega_{i,t}^{*} = \mu_{\omega^{*}}+\delta'\mbf{x}_{i, t-1}+ \varepsilon_{i}}
\end{frame}

% \subsection{Model Setup}

% % Technical Specifications
% \begin{frame}
% \frametitle{Technical Specifications (many, but familiar)}

% \end{frame}

% \subsection{Volatility Profile of a Time Series}
% \begin{frame}
% \frametitle{Volatilty Profile}
% In this particular setting, excess risk of an estimator $\theta$ has the form \\~\\

% \begin{align*}
%  R(\theta) & = \mathbb{E}_{x,y}[ (y-x^{T}\theta)^{2} -  (y-x^{T}\theta^{*})^{2}] \\
% & = (\theta - \theta^{*})^{T}\Sigma(\theta - \theta^{*})
% \end{align*}

% \end{frame}

% \begin{frame}
% \frametitle{What's the method here?}
%     \begin{alignat*}{12}
%     2 = 3
%     \end{alignat*}

% \end{frame}

% % Minimum Norm Estimator
% \begin{frame}
% \frametitle{Minimum Norm Estimator}

% \end{frame}

% \section{Post-shock Synthetic Volatility Forecasting Methodology}

% \begin{frame}
% \frametitle{Key Conceptual Innovation: Effective Rank}

% \end{frame}
% \section{Properties of Volatility Shock and Shock Estimators}

% \begin{frame}
% \frametitle{Main Result: Existence Proof, Dichotomy, and Bounds}

% \uncover<2-7>{Remarks}\\
% \begin{enumerate}
% \item<3-7> A gives us a (high probability) upper bound on the excess risk.

% \end{enumerate}
% \end{frame}

% \section{Numerical Examples}

% % So what do we want our eigenvalues to look like?
% \begin{frame}
% \fontsize{8pt}{9pt}
% \frametitle{So what do we want our eigenvalues to look like?}

% \end{frame}

% \begin{frame}
% \frametitle{Two Very Simple Examples}

% \end{frame}

% \begin{frame}
% \frametitle{A more benign example}

% \begin{example}[Coverging at the slowest rate possible]
% Fix $\alpha = 1, \beta > 1$.  Let $\lambda_{i} = \frac{1}{i \log^{\beta}(i+1)}$.

% \end{example}

% \end{frame}

% \begin{frame}
% \frametitle{What's the point of all this?}

% \end{frame}

% \begin{frame}
% \frametitle{How noise is hidden just right}

% \end{frame}

% \begin{frame}
% After all of this waiting, we formalize the notion under discussion.

% \begin{definition}[Asymptotically Benign]

% \end{definition}
% \end{frame}

% \section{Real Data Example}

% \begin{frame}
% \frametitle{Some things to think about with papers like this}

% \end{frame}

% \section{Discussion}

% \section{Future directions for Synthetic Volatility Forecasting}

% \subsection{Synthetic Impulse Response Functions}

% \section{Supplement}
% We analyze the real-world example with Brexit included.

\begin{frame}
    \frametitle{Bibliography}
    % https://latex.org/forum/viewtopic.php?t=13344
    % \bibliographystyle{plainnat}
    % \bibliography{synthVolForecast}

\printbibliography
\end{frame}

\end{document}