\documentclass[9pt]{beamer}

% Theme choice:
\usetheme{CambridgeUS}

%Packages
\usepackage[backend=biber,hyperref=true,doi=false,url=false,isbn=false, uniquename=false, uniquelist=false, style = authoryear-comp]{biblatex}
%https://www.overleaf.com/learn/latex/Biblatex_citation_styles
\addbibresource{synthVolForecast.bib}

%\usepackage{bibentry}

\usepackage{graphicx}
\usepackage{amsmath}
\usepackage{amsfonts}
\usepackage{amsthm}
\usepackage[export]{adjustbox}
\usepackage{amssymb}
\usepackage[useregional]{datetime2}
\usepackage{verbatim}
\usepackage{mathtools}% http://ctan.org/pkg/mathtools
\usepackage{mathrsfs}


\usepackage{amscd}
\usepackage{url}
% \usepackage[table,xcdraw,usenames]{xcolor}
% \usepackage[usenames]{color}

\usepackage{subcaption}
% \usepackage{enumitem}
% \usepackage{authblk}
% \usepackage{bm}
% \usepackage{pdfpages}

\usepackage{hyperref}
\usepackage{caption}
\usepackage{float}
%\usepackage[caption = false]{subfig}
\usepackage{tikz}
\usepackage{multirow}
\usepackage[linesnumbered, ruled,vlined]{algorithm2e}
\usepackage{pdflscape}
\usepackage{etoolbox}

%\AtBeginEnvironment{align}{\setcounter{equation}{0}} % https://tex.stackexchange.com/questions/349247/how-do-i-reset-the-counter-in-align

% function definition
\newcommand{\weight}{\pi}
\newcommand{\V}{\textbf{V}}
\newcommand{\ret}{\textbf{r}}
\newcommand{\y}{\textbf{y}}
\newcommand{\w}{\textbf{w}}
\newcommand{\x}{\textbf{x}}
\newcommand{\dbf}{\textbf{d}}
\newcommand{\X}{\textbf{X}}
\newcommand{\Y}{\textbf{Y}}
% \newcommand{\L}{\textbf{L}}
\newcommand{\Hist}{\mathcal{H}}
\newcommand{\Prob}{\mathbb{P}}
\def\mbf#1{\mathbf{#1}} % bold but not italic
\def\ind#1{\mathrm{1}(#1)} % indicator function
\newcommand{\simiid}{\stackrel{iid}{\sim}} %[] IID
\def\where{\text{ where }} % where
\newcommand{\indep}{\perp \!\!\! \perp } % independent symbols
\def\cov#1#2{\mathrm{Cov}(#1, #2)} % covariance
\def\mrm#1{\mathrm{#1}} % remove math
\newcommand{\reals}{\mathbb{R}} % Real number symbol
\def\t#1{\tilde{#1}} % tilde
\def\normal#1#2{\mathcal{N}(#1,#2)} % normal
\def\mbi#1{\boldsymbol{#1}} % Bold and italic (math bold italic)
\def\v#1{\mbi{#1}} % Vector notation
\def\mc#1{\mathcal{#1}} % mathical
\DeclareMathOperator*{\argmax}{arg\,max} % arg max
\DeclareMathOperator*{\argmin}{arg\,min} % arg min
\def\E{\mathbb{E}} % Expectation symbol
\def\mc#1{\mathcal{#1}}
\def\var#1{\mathrm{Var}(#1)} % Variance symbol
\def\checkmark{\tikz\fill[scale=0.4](0,.35) -- (.25,0) -- (1,.7) -- (.25,.15) -- cycle;} % checkmark
\newcommand\red[1]{{\color{red}#1}}
\def\bs#1{\boldsymbol{#1}}
\def\P{\mathbb{P}}
\def\var{\mathbf{Var}}
\def\naturals{\mathbb{N}}
\def\cp{\overset{p}{\to}}
\def\clt{\overset{\mathcal{L}^2}{\to}}

\setcounter{tocdepth}{4}
\setcounter{secnumdepth}{4}

\newcommand{\ceil}[1]{\lceil #1 \rceil}
\newcommand{\norm}[1]{\left\lVert#1\right\rVert} % A norm with 1 argument
\DeclareMathOperator{\Var}{Var} % Variance symbol

\newtheorem{cor}{Corollary}
\newtheorem{lem}{Lemma}
\newtheorem{thm}{Theorem}
\newtheorem{defn}{Definition}
\newtheorem{prop}{Proposition}
\theoremstyle{definition}
\newtheorem{remark}{Remark}
\hypersetup{
  linkcolor  = blue,
  citecolor  = blue,
  urlcolor   = blue,
  colorlinks = true,
} % color setup

% % \makeatletter
% % \setbeamertemplate{footline}
% % {
% %     \leavevmode%
% %     \hbox{%
% %         \begin{beamercolorbox}[wd=.333333\paperwidth,ht=2.25ex,dp=1ex,center]{author in head/foot}%
% %             \usebeamerfont{author in head/foot}\insertshortauthor
% %         \end{beamercolorbox}%
% %         \begin{beamercolorbox}[wd=.333333\paperwidth,ht=2.25ex,dp=1ex,center]{title in head/foot}%
% %             \usebeamerfont{title in head/foot}\insertshorttitle
% %         \end{beamercolorbox}%
% %         \begin{beamercolorbox}[wd=.333333\paperwidth,ht=2.25ex,dp=1ex,right]{date in head/foot}%
% %             \usebeamerfont{date in head/foot}\insertshortdate{}\hspace*{2em}
% %             \insertframenumber{} / \inserttotalframenumber\hspace*{2ex}
% %         \end{beamercolorbox}}%
% %         \vskip0pt%
% %     }
% %     \makeatother

\title{Synthetic Volatility Forecasting and Other Aggregation Techniques for Time Series Forecasting}
\subtitle{Preliminary Exam}
\author{David Lundquist\thanks{davidl11@ilinois.edu}, Daniel Eck\thanks{dje13@illinois.edu} (advisor)}
\date{April 10th, 2024}

\begin{document}

\part{one}
%% title frame
\begin{frame}
\titlepage
\end{frame}

\section{Introduction}

\begin{frame}
\frametitle{A seemingly unprecedented event might make one ask}
\begin{enumerate}
    \item What does it resemble from the past?
    \item What past events are most relevant?
    \item Can we incorporate past events in a systematic, principled manner?
\end{enumerate}
\end{frame}

\begin{frame}
    \frametitle{When would we ever have to do this?}

    \begin{itemize}
        \item Event-driven investing strategies (unscheduled news shock)
        \item Pairs trading strategies
        \item Structural shock to macroeconomy (scheduled news possibly pre-empted by additional news)
        

        \href{https://www.wsj.com/articles/bad-inflation-reports-raise-odds-of-surprise-0-75-percentage-point-rate-rise-this-week-11655147927}{\includegraphics[scale=.3]{WSJ_rate_hike_2022.png}}
    \end{itemize}
\end{frame}

\begin{frame}

    \begin{example}[Weekend of March 6th - 8th, 2020]
        \href{https://www.governor.ny.gov/news/novel-coronavirus-briefing-governor-cuomo-declares-state-emergency-contain-spread-virus}{\includegraphics[scale=.3]{NYS_state.png}}

        \href{https://www.cnbc.com/2020/03/08/opec-deal-collapse-sparks-price-war-20-oil-in-2020-is-coming.html}{\includegraphics[scale=.3]{cnn.png}}

        \href{https://www.cnn.com/2020/03/08/investing/oil-prices-crash-opec-russia-saudi-arabia/index.html}{\includegraphics[scale=.3]{cnbc.png}}
        \end{example}

\end{frame}

\begin{frame}
\frametitle{Punchline of the paper}

\begin{itemize}
    \item <1-> Credible forecasting is possible under news shocks, so long as we incorporate external information to account for the nonzero errors.
    \item <2-> \begin{figure}[H]
        \begin{center}
          \includegraphics[scale=.3]{simulation_plots/alternative_USE_in_paper_simulation_plot_arithmetic_mean.png}
          \caption{Predicting Volatility Using Only Arithmetic Mean of Donors}
          \end{center}
        \end{figure}
\end{itemize}
\end{frame}


\begin{frame}
    \frametitle{Background and related methods}
    Volatility Modeling

    \begin{itemize}
        \item GARCH is slow to react to shocks \parencite[][]{andersen2003modeling}
        \item Asymmetric GARCH models catch up faster but need post-shock data
        \item Realized GARCH \parencite[][]{hansen2012realized}, in our setting, would require post-shock information and/or high-frequency data in order to outperform, and Realized GARCH is highly parameterized
    \end{itemize}
\end{frame}

\begin{frame}
    \frametitle{Background and related methods}

    Forecast Augmentation
    \begin{itemize}
        \item \cite[][]{clements1996intercept,clements1998forecasting} laid the groundwork for modeling nonzero errors in time series forecasting
        \item \cite[][]{guerron2017macroeconomic} use a series' own errors to correct the forecast for that series
        \item \cite[][]{dendramis2020similarity} use a similarity-based procedure to correct linear parameters in time series forecasts
        \item \cite[][]{foroni2022forecasting} adjust pandemic-era forecasts using intercept correction techniques and data from Great Financial Crisis
        \item \cite[][]{lin2021minimizing} use distanced-based weighting (a similarity approach) to aggregate and weight fixed effects from a donor pool
    \end{itemize}
\end{frame}

\part{two}
\section{Setting}
% Outline frame
\begin{frame}{Outline} % https://tex.stackexchange.com/questions/73196/beamer-table-of-contents-shade-all-previous-sections
    \tableofcontents[part=1,currentsection]\tableofcontents
\end{frame}

% Presentation structure

% Setting for the problem
\begin{frame}
\frametitle{Premise: News has broken but markets are closed}

\begin{itemize}
\item After-hours trading provides a poor forum in which to digest news
\item The news constitutes public, material information relevant to one or more traded assets
\item The qualitative aspects of the news provide a basis upon which to 
\begin{itemize}
    \item match to past events
    \item match in a $p$-dimensional covariate space
\end{itemize}
\end{itemize}
\end{frame}

\begin{frame}
    \frametitle{A Primer on GARCH}

    Let $\{a_{t}\}$ denote an observable, real-valued discrete-time stochastic process.\\
    

    We say $\{a_{t}\}$ is a strong GARCH process with respect to $\{\epsilon_{t}\}$ iff 
    \begin{align*}
        &\sigma_{t}^{2} = \omega + \sum^{m}_{k=1}\alpha_{k}a^{2}_{t-k} + \sum_{j=1}^{s}\beta_{j}\sigma_{t-j}^{2}\\
        &a_{t} = \sigma_{t}\epsilon_{t}\\
        &\epsilon_{t} \simiid E[\epsilon_{t}]=0, Var[\epsilon_{t}] = 1\\
        &\forall k,j, \alpha_{k},\beta_{j}\geq 0\\ 
        &\forall t, \omega, \sigma_{t} > 0 
        \end{align*}
\end{frame}

\begin{frame}
    \frametitle{Volatility Equation with an exogenous term: GARCH-X}
    
    \begin{align*}
        &\sigma_{t}^{2} = \omega+ \sum^{m}_{k=1}\alpha_{k}a^{2}_{t-k} + \sum_{j=1}^{s}\beta_{j}\sigma_{t-j}^{2} + \gamma^{T}\x_{t} \text{ .}\label{GARCH-X}
    \end{align*}
    
    \end{frame}
    
\begin{frame}
    \frametitle{Model Preliminaries}

    \fontsize{8}{7.2}

    Let $I(\cdot)$ be an indicator function.  \\
\bigbreak
    Let $T_i$ denote the time length of the time series $i$ for $i = 1, \ldots, n+1$.\\
    \bigbreak
    Let $T_i^*$ denote the largest time index prior to news shock, with $T_i^* < T_i$. \\
    \bigbreak
    Let $\delta, \x_{i,t} \in \mathbb{R}^{p}$.  
\end{frame}

\begin{frame}
\frametitle{Model Setup}

\fontsize{6}{7.2}

For $t= 1, \ldots, T_i$ and $i = 1, \ldots, n+1$, the model $\mc{M}_1$ is defined as 
\begin{align*}
  \mc{M}_1 \colon \begin{array}{l}
     \sigma^{2}_{i,t} = \omega_{i} + \omega^{*}_i + \sum^{m_{i}}_{k=1}\alpha_{i,k}a^{2}_{i,t-k} + \sum_{j=1}^{s_{i}}\beta_{i,j}\sigma_{i,t-j}^{2} + \gamma_{i}^{T} \x_{i,t} \text{ }\\[.2cm]
     a_{i,t} = \sigma_{i,t}((1-D^{return}_{i,t})\epsilon_{i,t} + D^{return}_{i,t}\epsilon^{*}_{i})\\[.2cm]
    \omega_{i,t}^{*} = D^{vol}_{i,t}[\mu_{\omega^{*}}+\delta'\mbf{x}_{i, t-1}+ u_{i,t}],
  \end{array}
  \end{align*}\label{model_1}
with error structure

  \begin{align*}
    \epsilon_{i,t} &\simiid \mc{F}_{\epsilon} \text{ with }  \; \mrm{E}_{\mc{F}_{\epsilon}}(\epsilon) = 0, \mrm{Var}_{\mc{F}_{\epsilon}}(\epsilon)  = 1  \\
    \epsilon^{*}_{i,t} &\simiid \mc{F}_{\epsilon^{*}} \text{ with }  \; \mrm{E}_{\mc{F}_{\epsilon^{*}}}(\epsilon) = \mu_{\epsilon^{*}}, \mrm{Var}_{\mc{F}_{\epsilon^{*}}}(\epsilon^{*})  = \sigma^2_{\epsilon^{*}}  \\
    u_{i,t} & \simiid  \mc{F}_{u} \text{ with }  \; \mrm{E}_{\mc{F}_{u}}(u) = 0, \mrm{Var}_{\mc{F}_{u}}(u) = \sigma^2_{u}\\
    \epsilon_{i,t} & \indep  \epsilon^{*}_{i,t}  \indep u_{i,t}
    \end{align*}
where $D^{return}_{i,t} = I(t \in \{T_i^* + 1,...,T_i^* + L_{i, return}\})$ and $D^{vol}_{i,t} = I(t \in \{T_i^* + 1,...,T_i^* + L_{i, vol}\})$ and $L_{i,return},L_{i,vol}$ denote lengths of log return and volatility shocks, respectively.  
\end{frame}

\begin{frame}
    \frametitle{Model Details}
    Let $\mc{M}_{0}$ denote the subclass of $\mc{M}_{1}$ models such that $\delta \equiv 0$.  \\
    \bigbreak
    Note that $\mc{M}_{0}$ assumes that $\omega^{*}_i$ have no dependence on the covariates and are i.i.d. with $\E[ \omega^{*}_i]=\mu_{\omega^{*}}$.  
\end{frame}

\begin{frame}
\frametitle{Our Model is Nested inside a Factor Model}

\fontsize{6}{7.2}

Consider $\mc{M}_{1}$ in the context of the factor model from \cite[][]{abadie2010synthetic}, where an untreated unit is governed by:

$$Y^{N}_{i,t} = \delta_{t} + \boldsymbol\theta_{t}\textbf{Z}_{i}+\boldsymbol\lambda_{t}\boldsymbol\mu_{i}+\varepsilon_{i,t}$$

which nests the GARCH model's volatilty equation as well as the ARMA representation of a GARCH model, where

\begin{align*}
\delta_{t} & \sim \omega, \text{a location parameter shared across donors}\\
\boldsymbol\theta_{t} & \sim \boldsymbol\alpha_{k}, \text{a vector of ARCH parameters and other coefficients shared across donors} \\
\textbf{Z}_{i} & \sim \boldsymbol a_{i,t-k}, \text{a vector of observable quantities specific to each donor} \\
\boldsymbol \lambda_{t} & \sim \boldsymbol\beta_{j}, \text{a vector of GARCH parameters shared across donors} \\
\boldsymbol \mu_{i} & \sim \boldsymbol \sigma_{i,t-j}^{2}, \text{a vector of latent quantities specific to each donor}   \\
\end{align*}

% and $\varepsilon_{i,t}$ is idiosyncratic noise, uncorrelated across time and donors.
\end{frame}


\begin{frame}
\frametitle{Volatility Profile of a Time Series}
\fontsize{6.6}{7}

\begin{equation*}
    \V_{t} = 
    \begin{pmatrix}
    \hat\alpha_{1,t} & \hat\alpha_{t,2}  & \cdots & \hat\alpha_{t,n}  \\
    \hat\beta_{1,t} & \hat\beta_{t,2}  & \cdots & \hat\beta_{t,n}  \\
    \vdots  & \vdots  & \ddots & \vdots  \\
    RV_{1,t} & RV_{2,t}  & \cdots & RV_{n,t}  \\
    RV_{1,t-1}  & RV_{2,t-1}  & \cdots & RV_{n,t-1}  \\
    \vdots  & \vdots  & \ddots & \vdots  \\
    IV_{1,t} & IV_{2,t} & \cdots & IV_{n,t} \\
    IV_{1,t-1}  & IV_{2,t-1}  & \cdots & IV_{n,t-1} \\
    \vdots  & \vdots  & \ddots & \vdots  \\
    |r_{1,t}| & |r_{2,t}| & \cdots & |r_{n,t}| \\
    |r_{1,t-1}|  & |r_{2,t-1}|  & \cdots & |r_{n,t-1}| \\
    \end{pmatrix},
    \end{equation*}
    \bigbreak
    where RV denotes realized variance and IV the implied volatility
\end{frame}

\begin{frame}
    \frametitle{Significance of the Volatility Profile}
    Covariates chosen for inclusion in a given volatility profile may be any $\mathcal{F}_{t}$-measurable function, for example
    \begin{itemize}
        \item levels
        \item differences in levels
        \item log returns
        \item percentage returns
        \item absolute values of the above
    \end{itemize}

    \bigbreak 

    Key criterion for inclusion: how plausible is the covariate as a proxy for risk conditions?
\end{frame}

\section{Post-shock Synthetic Volatility Forecasting Methodology}

\begin{frame}
\frametitle{Forecasting}

\fontsize{7.6}{7}

We present two forecasts:

\begin{align*}
  \text{Forecast 1: } & \hat\sigma^{2}_{unadjusted} = \hat\E_{T^{*}}[\sigma^{2}_{1,T_{1}^{*}+1}|\mathcal{F}_{T^{*}}] = \hat\omega_{i} + \sum^{m_{i}}_{k=1}\hat\alpha_{i,k}a^{2}_{i,t-k} + \sum_{j=1}^{s_{i}}\hat\beta_{i,j}\sigma_{i,t-j}^{2} + \hat\gamma_{i}^{T} \x_{i,t}\\
  \text{Forecast 2: } & \hat\sigma^{2}_{adjusted} = \hat\E_{T^{*}}[\sigma^{2}_{1,T_{1}^{*}+1}|\mathcal{F}_{T^{*}}] + \hat\omega^{*} = \hat\omega_{i} + \sum^{m_{i}}_{k=1}\hat\alpha_{i,k}a^{2}_{i,t-k} + \sum_{j=1}^{s_{i}}\hat\beta_{i,j}\sigma_{i,t-j}^{2} + \hat\gamma_{i}^{T} \x_{i,t} + \hat\omega^{*} \text{ .}
\end{align*}
\end{frame}

\begin{frame}
\frametitle{Excess Volatility Estimators}

Observe the pair $(\{\hat\omega^{*}_{i}\}^{n+1}_{i=2},\{\textbf{v}_{i}\}^{n+1}_{i=2})$.  \\

\bigbreak

Goal: recover weights $\{\weight_{i}\}^{n+1}_{i=2} \in \Delta^{n}$ and compute $\hat\omega^{*} \coloneq \sum^{n+1}_{i=2}\weight_{i}\hat\omega^{*}_{i}$, our forecast adjustment term.

\bigbreak

Following \cite[][]{abadie2003economic,abadie2010synthetic}, let $\|\cdot\|_{\textbf{S}}$ denote any semi-norm on $\mathbb{R}^{p}$, and define

\begin{align*}
\{\pi\}_{i=2}^{n+1} = \argmin_{\pi}\|\textbf{v}_{1,T^{*}} - \V_{T^{*}}\pi \|_{\textbf{S}} \text{ .}
\end{align*}

Nota bene: the weights $\{\weight_{i}\}_{i=2}^{n+1}$ are deterministic with respect to $\mathcal{F}_{T^{*}}$.

\end{frame}

\begin{frame}
\frametitle{Ground Truth Estimators}

We use realized volatility (RV)
\begin{itemize}
\item ``model-free'' in the sense that it requires no modeling assumptions \parencite[][]{andersen2010stochastic}.  
\item RV can be decomposed into the sum of a continous component and a jump component, with the latter being less predictable and less persistent \parencite[][]{andersen2007roughing}, cited in \cite[][]{de2006forecasting}, two factors that further motivate our method.
\end{itemize}

\end{frame}

\begin{frame}
    \frametitle{Realized Volatility Estimation}

Examine $K$ units of of time; each unit is divided into $m$ intervals of length $\frac{1}{m}$.  Let $p_{t} = \log{P_{t}}$, and let $\tilde{r}(t,\frac{1}{m}) = p_{t} - p_{t-\frac{1}{m}}$  \parencite[][]{andersen2009realized}. 

\bigbreak

Estimate variance of $i$th log return series using Realized Volatility of the $K$ consecutive trading days that conclude with day $t$, denoted $RV_{i,t}^{K,m}$, using

$$RV_{i,t}^{K,m} = \frac{1}{K}\sum^{Km}_{v=1}\tilde{r}^{2}(v/m,1/m),$$

where the $K$ trading days have been chopped into $Km$ equally-sized blocks.

Assuming the $K$ units $\tilde{r}(t, 1) = p_{t} - p_{t-1}$ are s.t. $\tilde{r}(t, 1) \simiid N(\mu, \delta^{2})$, it is easily verified that 

$$\E[RV^{K,m}] = \frac{\mu^{2}}{m} + \delta^{2},$$
which is a biased but consistent estimator of the variance.  We pick $m = 77$, corresponding to the 6.5-hour trading day chopped into 5-minute blocks, omitting first five-minutes of the day.

\end{frame}

\begin{frame}
\frametitle{Loss Functions}

Aim: point forecasts for $\sigma^{2}_{1,T^{*}+h}|\mathcal{F}_{T^{*}}$, $h=1,2,...,$, the $h$-step ahead conditional variance for the time series under study\\

\bigbreak
 
Let $L^{h}$ with the subscripted pair $\{$prediction method, ground truth estimator$\}$, denote the loss function for an $h$-step-ahead forecast using a given prediction function and ground truth estimator.  

\end{frame}

\begin{frame}
    \frametitle{Loss Function Examples}
For example, one loss function of interest in this study is the 1-step-ahead MSE using Synthetic Volatility Forecasting and Realized Volatility:

$$\text{MSE}^{1}_{\text{SVF, RV}} = (\hat\sigma^{2}_{SVF} - \hat\sigma^{2}_{RV})^{2}$$
Also of interest in mean absolute percentage error for an $h$-step-ahead forecast, defined as

$$\text{MAPE}^{h}_{method, ground truth} = \frac{|\hat\sigma^{2}_{h, method} - \hat\sigma^{2}_{h, ground truth}|}{\hat\sigma^{2}_{h, ground truth}}$$

\end{frame}

\begin{frame}\frametitle{Our choice of Loss Function}

    Finally, we introduce the QL (quasi-likelihood) Loss \parencite[][]{brownlees2011practical}:

    $$\text{QL}^{h}_{method, ground truth} = \frac{ \hat\sigma^{2}_{h, method} }{\hat\sigma^{2}_{h, ground truth}} - \log{\frac{ \hat\sigma^{2}_{h, method} }{\hat\sigma^{2}_{h, ground truth}}} -1 \text{ .}$$
    
What distinguishes QL Loss? \\

\begin{itemize}

\item Multiplicative rather than additive

\item This has benefits, both practical and theoretical.  

\item As \cite[][]{brownlees2011practical} explain, ``[a]mid volatility turmoil, large MSE
losses will be a consequence of high volatility without necessarily corresponding to deterioration of forecasting ability. The QL avoids this ambiguity, making it easier to compare losses across volatility regimes."

\end{itemize}

\end{frame}

\section{Properties of Volatility Shock and Shock Estimators}\label{SVF_properties}
\frametitle{Two Consistency Results}

\section{Real Data Example}

\begin{frame}{Why apply our method to the 2016 US Election?}
    
\end{frame}
\begin{frame}
    \begin{figure}[H]
        \begin{center}
          \includegraphics[scale=.2]{iyg.png}
          \caption{Includes: JPM, BAC, WF, among other financial majors}
          \end{center}
        \end{figure}
\begin{enumerate}
    \item \textbf{Model choice} GARCH(1,1) on the daily log return series of IYG in each donor

    \item \textbf{Covariate Choice} log return Crude Oil (CL.F), the VIX (VIX) and the log return of the VIX, the log returns of the 3-month, 5-year, 10-year, and 30-year US Treasuries, return of the most recently available monthly spread between AAA and BAA corporate debt, widely considered a proxy for lending risk \parencite[][]{goodell2013us, kane1996p}.  We also include the log return in the trading volume of the ETF IYG itself, which serves as a proxy for panic.

    \item \textbf{Donor pool construction} the three most recent US presidential elections prior to the 2016 election.  The three US presidential elections are the only presidential elections since the advent of the ETF IYG.  We exclude the midterm congressional elections in the US, which generate far lower voter turnout and feature no national races.

    \item \textbf{Choice of estimator for volatility} Sum of squared 5-minute log returns of IYG on November 9th, 2016, otherwise known as the Realized Volatility estimator of volatility \parencite[][]{andersen2009realized}, as our proxy.  We exclude the first five minutes of the trading day, resulting in a sum of 77 squared five-minute returns generated between 9:35am and 4pm.
\end{enumerate} 

\end{frame}

\begin{frame}{2016 Election}
    \begin{figure}[H]
        \begin{center}
          \includegraphics[scale=.3]{real_data_output_plots/savetime_SunMar172252462024_IYG_CL=F-^VIX-^IRX-^FVX-^TNX-^TYX_^VIX_2016-11-08-2004-11-02-2008-11-04-2012-11-06.png}
          \caption{The volatility induced by the 2016 US election}
          \label{fig:SVF_2016}
          \end{center}
        \end{figure}
\end{frame}

\section{Numerical Examples}

% So what do we want our eigenvalues to look like?
\begin{frame}
\fontsize{8pt}{9pt}

\frametitle{Simplest Simulation Setup}

Most elementary simulation uses $\mc{M}_1$ and varies only \textcolor{blue}{two} parameters.\\

\bigbreak

Recall an \hyperlink{model_1}{$\mc{M}_1$} model on the volatility, which is characterized by an exogenous shock to the volatility equation generated by an affine function of the covariates:

  \begin{align*}
    \mc{M}_1 \colon \begin{array}{l}
       \sigma^{2}_{i,t} = \omega_{i} + \omega^{*}_{i,t} + \sum^{m_{i}}_{k=1}\alpha_{i,k}a^{2}_{i,t-k} + \sum_{j=1}^{s_{i}}\beta_{i,j}\sigma_{i,t-j}^{2} + \gamma_{i}^{T} \x_{i,t} \text{ }\\[.2cm]
       a_{i,t} = \sigma_{i,t}((1-D^{return}_{i,t})\epsilon_{i,t} + D^{return}_{i,t}\epsilon^{*}_{i})\\[.2cm]
      \omega_{i,t}^{*} = D^{vol}_{i,t}[\mu_{\omega^{*}}+\delta'\mbf{x}_{i, t}+ u_{i,t}]\\[.2cm]
      D^{return}_{i,t} \equiv 0
    \end{array}
    \end{align*}

% https://www.overleaf.com/learn/latex/Beamer_Presentations%3A_A_Tutorial_for_Beginners_(Part_3)%E2%80%94Blocks%2C_Code%2C_Hyperlinks_and_Buttons
\end{frame}

\begin{frame}
    \fontsize{8pt}{9pt}
    What do we vary?\\

    \bigbreak
    
    The volatility shock signal $\delta$ and the variance of volatility shock noise $u_{t}$\\
    
    \begin{figure}[h!]
      \begin{center}
        \includegraphics[scale=.29]{simulation_plots/standard_simulation_alpha_.1_beta_.82.png}
        \caption{Fixed parameter values: $\alpha = .1, \beta = .82, \mu_{x} = 1, \sigma_{x} = .1$}\label{fig:heavy_beta}
      \end{center}
      \end{figure}

\end{frame}

\begin{frame}
    \fontsize{8pt}{9pt}

    If we switch the values of $\alpha$ and $\beta$, we see similar behavior. 
    \begin{figure}[h!]
      \begin{center}
        \includegraphics[scale=.29]{simulation_plots/standard_simulation_alpha_.82_beta_.1.png}
        \caption{Fixed parameter values: $\alpha = .82, \beta = .1, \mu_{x} = 1, \sigma_{x} = .1$}
        \label{fig:heavy_alpha}
      \end{center}
      \end{figure}

\end{frame}

\section{Discussion}

\section{Future directions for Synthetic Volatility Forecasting}

\begin{frame}
We shall group the extensions into five buckets:
\begin{itemize}
    \item How much can we automate?
    \item Alternatives for fixed effect estimation
    \item Alternative estimators and estimands
    \item What can you do with a volatility forecast?
    \item Where else is distanced-based weighting useful?
\end{itemize}
\end{frame}

\begin{frame}{How much can we automate?}

    
\end{frame}

\begin{frame}
\frametitle{Alternative Ways of Estimating Fixed Effects}
High-frequency data?

\begin{itemize}

\item{Realized GARCH with High-Frequency Data}

\item{Stochastic Volatility}
\end{itemize}
\end{frame}

\begin{frame}
    \frametitle{Alternative Estimators and Estimands in Volatility Modeling}
    \begin{itemize}
        
        \item Factors in volatility profile
        \item Overnight returns instead of open-to-close
        
        \item Signal Recovery Perspective \parencite{ferwana2022optimal}
        
        \item Stochastic Volatility: Correlation between errors
        \item Multivariate GARCH
        
        \end{itemize}
\end{frame}

\begin{frame}
    \frametitle{What can you do with a volatility forecast?}
    \begin{itemize}
        \item{Value-at-Risk using SVF-based $\hat\sigma^{2}_{t}$}
        \end{itemize}
\end{frame}

\begin{frame}
    \frametitle{New Frontiers in Distance-based Weighting}
    \begin{itemize}
        \item Integrate lessons from literature on under/over reactions to information shocks \parencite[][]{jiang2017information}
        \item{Synthetic Impulse Response Functions}
        \end{itemize}
\end{frame}

\begin{frame}{Synthetic Impulse Response Functions: A Proposal}
    \begin{itemize}
        \item Suppose we have a multivariate time series of dimension $p \times T$ subject to shocks from a common shock distribution
        \item Using an IRF estimate aggregated from the first $n$ shocks of interest, we predict the response of variable $i$ from variable $j$, $1\leq i \leq j \leq p$. 
    \end{itemize}
    
\end{frame}
\section{Supplement}
We analyze the real data example with Brexit included.

\begin{figure}[H]
    \begin{center}
      \includegraphics[scale=.32]{real_data_output_plots/savetime_SunMar172251232024_IYG_6B=F-CL=F-^VIX-^IRX-^FVX-^TNX-^TYX_^VIX_2016-11-08-2004-11-02-2008-11-04-2012-11-06-2016-06-22.png}
      \end{center}
    \end{figure}
    

\begin{frame}[t,allowframebreaks]
    \frametitle{References}
    % https://latex.org/forum/viewtopic.php?t=13344
    % \bibliographystyle{plainnat}
    % \bibliography{synthVolForecast}

\printbibliography[heading=none]
\end{frame}

\end{document}