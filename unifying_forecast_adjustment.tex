\documentclass[11pt]{article}

\pdfminorversion=4

% use packages
\usepackage[utf8]{inputenc}
\usepackage{amsmath}
\usepackage{amsthm}
\usepackage{amsfonts}
\usepackage{amscd}
\usepackage{amssymb}
\usepackage{natbib}
\usepackage{url}
\usepackage[table,xcdraw,usenames]{xcolor}
%\usepackage[usenames]{color}

\usepackage{graphicx}
\usepackage{subcaption}
\usepackage{mathtools}
\usepackage{enumitem}
\usepackage{authblk}
\usepackage{bm}
\usepackage{comment}
\usepackage{pdfpages}

\usepackage{hyperref}
\usepackage{caption}
\usepackage{float}
%\usepackage[caption = false]{subfig}
\usepackage{tikz}
\usepackage{multirow}
\usepackage[linesnumbered, ruled,vlined]{algorithm2e}
\usepackage{pdflscape}
\usepackage{etoolbox}

%\AtBeginEnvironment{align}{\setcounter{equation}{0}} % https://tex.stackexchange.com/questions/349247/how-do-i-reset-the-counter-in-align

% margin setup
\usepackage{geometry}
\geometry{margin=0.8in}

% function definition
\newcommand{\V}{\textbf{V}}
\newcommand{\weight}{\pi}
\newcommand{\ret}{\textbf{r}}
\newcommand{\y}{\textbf{y}}
\newcommand{\w}{\textbf{w}}
\newcommand{\x}{\textbf{x}}
\newcommand{\dbf}{\textbf{d}}
\newcommand{\X}{\textbf{X}}
\newcommand{\Y}{\textbf{Y}}
%\newcommand{\L}{\textbf{L}}
\newcommand{\Hist}{\mathcal{H}}
\newcommand{\Prob}{\mathbb{P}}
\def\mbf#1{\mathbf{#1}} % bold but not italic
\def\ind#1{\mathrm{1}(#1)} % indicator function
\newcommand{\simiid}{\stackrel{iid}{\sim}} %[] IID 
\def\where{\text{ where }} % where
\newcommand{\indep}{\perp \!\!\! \perp } % independent symbols
\def\cov#1#2{\mathrm{Cov}(#1, #2)} % covariance 
\def\mrm#1{\mathrm{#1}} % remove math
\newcommand{\reals}{\mathbb{R}} % Real number symbol
\def\t#1{\tilde{#1}} % tilde
\def\normal#1#2{\mathcal{N}(#1,#2)} % normal
\def\mbi#1{\boldsymbol{#1}} % Bold and italic (math bold italic)
\def\v#1{\mbi{#1}} % Vector notation
\def\mc#1{\mathcal{#1}} % mathical
\DeclareMathOperator*{\argmax}{arg\,max} % arg max
\DeclareMathOperator*{\argmin}{arg\,min} % arg min
\def\E{\mathbb{E}} % Expectation symbol
\def\mc#1{\mathcal{#1}}
\def\var#1{\mathrm{Var}(#1)} % Variance symbol
\def\checkmark{\tikz\fill[scale=0.4](0,.35) -- (.25,0) -- (1,.7) -- (.25,.15) -- cycle;} % checkmark
\newcommand\red[1]{{\color{red}#1}}
\def\bs#1{\boldsymbol{#1}}
\def\P{\mathbb{P}}
\def\var{\mathbf{Var}}
\def\naturals{\mathbb{N}}
\def\cp{\overset{p}{\to}}
\def\clt{\overset{\mathcal{L}^2}{\to}}

\setcounter{tocdepth}{4}
\setcounter{secnumdepth}{4}

\newtheorem{corollary}{Corollary}
\newcommand{\ceil}[1]{\lceil #1 \rceil}
\newcommand{\norm}[1]{\left\lVert#1\right\rVert} % A norm with 1 argument
\DeclareMathOperator{\Var}{Var} % Variance symbol

\newtheorem{cor}{Corollary}
\newtheorem{lem}{Lemma}
\newtheorem{thm}{Theorem}
\newtheorem{defn}{Definition}
\newtheorem{prop}{Proposition}
\theoremstyle{definition}
\newtheorem{remark}{Remark}
\hypersetup{
  linkcolor  = blue,
  citecolor  = blue,
  urlcolor   = blue,
  colorlinks = true,
} % color setup

% proof to proposition 
\newenvironment{proof-of-proposition}[1][{}]{\noindent{\bf
    Proof of Proposition {#1}}
  \hspace*{.5em}}{\qed\bigskip\\}
% general proof of corollary
  \newenvironment{proof-of-corollary}[1][{}]{\noindent{\bf
    Proof of Corollary {#1}}
  \hspace*{.5em}}{\qed\bigskip\\}
% general proof of lemma
  \newenvironment{proof-of-lemma}[1][{}]{\noindent{\bf
    Proof of Lemma {#1}}
  \hspace*{.5em}}{\qed\bigskip\\}

\allowdisplaybreaks

\title{Forecast Adjustment Under Shocks: A Unification}
\author{David Lundquist\thanks{davidl11@ilinois.edu}, Daniel Eck\thanks{dje13@illinois.edu} }
\affil{Department of Statistics, University of Illinois at Urbana-Champaign}
\date{\today}

\begin{document}

\maketitle

\begin{abstract} 
This work systematizes and unifies the rich landscape of model adjustment and model correction methods, with a special focus on forecast adjustment under the presence of shocks.


\end{abstract}

\section{Introduction}

Forecasting amid anticipated shocks raises unavoidable questions: should the forecast model be abandoned in favor of a discretionary or ad-hoc or one-off adjustment?  Does the does the discretion of a forecaster rule out a quantitative method for making the adjustment?  What is the ultimate purpose of the adjustment, and how it is to be used?

This work aims to systemaize and unify a range of conceptual approaches and tools that have developed across the broad ecosystem of the econometric and forecasting literatures.\\

difference between discretionary and automated\\

Setting the model back on track\\

what we are talking about here is not forecast combination, but there may be, nevertheless, a role for forecast combination: combining the forecasts generated by small differences in covariate and/or donor choice\\

The role and meaning of similarity\\

How important is a shared DGP?

\subsection{Model Adjustment Using Similarity-Based Parameter Correction: A Global Overview}

\begin{enumerate}
  \item a random object to forecast that depends on a linear specification
  \item a parametric model family shared by donors
  \item a correction term for the model family shared by donors
  \item a parametric specification for the correction term

  \item a reliable estimation procedure for the shared model
  \begin{enumerate}
  \item This should be straightforward
  \end{enumerate}
  \item a reliable estimation procedure for the correction term 
  \begin{enumerate}
    \item This might not be straightforward.  Some models like GARCH, for example, might deliver very noisy estimates for indicator variables that occur just once.
    \end{enumerate}
  \item a correction function that aggregates (i.e. maps) donor correction terms based on some notion of similarity
\end{enumerate}

\section{Setting}

\section{Model-Specific Considerations}

\subsection{ARIMA}
\subsection{GARCH}
\subsection{HAR}
\subsection{VAR}

\section{Real Data Examples}
\section{Discussion}

\begin{itemize}
  \item Binary Outcome Forecasts
  \item Density Forecasts
  \item Quantile Forecasts
\end{itemize}

\clearpage

\bibliographystyle{plainnat}
\bibliography{synthVolForecast}
 
\end{document}